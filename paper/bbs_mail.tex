\documentclass{article}
\usepackage{amsfonts}
\usepackage[parfill]{parskip}

\begin{document}

...

2) The existence of the formula follows from general theory that long predates the DeepSets paper. This seems worth mentioning.

2a) It's known classically (by the late 18th century) that permutation-invariant polynomials on a set of 1-dimensional data are representable in terms of (in fact, are polynomials in) the power sums $\sum_i x_i^j$ for $j=1,\dots,n$ (where $n$ is the size of the data). The first written proof on the record is probably the one in Eduard Waring’s \textit{Meditationes Algebraicae} (1782).

2b) Weyl proved in his \textit{The Classical Groups} (the edition everybody cites is 1946 but I think there was an earlier edition?) that the permutation-invariant polynomials on a set of any-dimensional data are representable in terms of (in fact, polynomials in) the polarizations of the permutation-invariant polynomials on a corresponding set of 1-dimensional data. 

Your $F_{mn}$'s are the 2-dimensional polarizations of the 1-dimensional power sums. So the existence of your formula follows from (2a) and (2b).

...

[D]efinition: if $f$ is a polynomial function of a column vector $v\in \mathbb{R}^n$, then the \textit{$d$-dimensional polarizations} of $f$ are functions on $d$ vectors $v_1,\dots,v_d\in \mathbb{R}^{n\times d}$ of the coefficients of each monomial in $a_1,\dots,a_d$ in the expansion of $f(a_1v_1 + \dots + a_dv_d)$.
In your setting, $v_1$ is $(x_1,x_2,x_3)^T$ and $v_2$ is $(y_1,y_2,y_3)^T$, $F_{10}$ and $F_{01}$ are the polarizations of the first power sum, and $F_{20}$, $F_{11}$, and $F_{02}$ are the polarizations of the second power sum.


\end{document}