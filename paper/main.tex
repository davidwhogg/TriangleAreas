\documentclass[12pt]{article}
\usepackage[letterpaper]{geometry}
\usepackage{amsmath}

\frenchspacing
\begin{document}

\section*{\raggedright
Another formula for the area of a triangle: Useless, but explicitly in map-reduce format}
\noindent \textbf{David W. Hogg} and \textbf{Soledad Villar}

\paragraph{Abstract:}
It is known that any permutation-invariant function of data points $x_i$
can be written in the form $\rho(\sum_i\phi(x_i))$,
where $\rho$ and $\phi$ are (probably nonlinear) functions, and the function $\phi$ might output high-dimensional output (but can't depend on $i$).
This form can be considered map--reduce, because the $\phi$ executions can be performed asynchronously and the sums can be performed hierarchically.
The area of a triangle is a permutation-invariant function of the locations of the three corners.
We find (embarrassingly, by symbolic regression) the formula for the area of the triangle that is explicitly in this map--reduce form.
It is ugly.

\section{Introduction}
The area $A$ of a triangle can be written in many ways:
\begin{align}
A &= \frac{1}{2}\,\text{(base)}\,\text{(height)} \\
  &= \frac{1}{2}\,|(\vec{x}_3 - \vec{x}_1) \times (\vec{x}_2 - \vec{x}_1)| \\
  &= \frac{1}{2}\,a\,b\,\sin(C) \\
  &= \sqrt{s\,(s-a)\,(s-b)\,(s-c)} \label{eq:Heron} ~,
\end{align}
where the first formula is the usual introduction to the subject,
$(\vec{x}_1, \vec{x}_2, \vec{x}_3$ are three vectors pointing from an origin to the corners of the triangle,
$\times$ represents the cross-product operator (or its generalization to higher dimensions),
$a, b, c$ are side lengths of the triangle,
$C$ is the angle between the $a$ and $b$ sides,
and $s$ is the semi-perimeter $(a + b + c)/2$.

While the area of a triangle does not depend on how its sides or corners are labeled,
only the last of these formulae---Heron's formula \eqref{eq:Heron}---is explicitly permutation-invariant.
The theory of permutation-invariant functions says (CITE) that any per\-mu\-tation-invariant function (under conditions??) of $N$ objects $x_i$ (with $1\leq i\leq N$) can be approximated (to arbitrary precision??) with an expression of the form
\begin{equation}
    \rho(\sum_{i=1}^N\phi(x_i)) \label{eq:DeepSets}
\end{equation}
where $\rho$ and $\phi$ are (possibly very nonlinear) functions,
and the $x_i$ are the data points that can be permuted without changing the result.
This result---which we associate with the subject \emph{Deep Sets} (CITE)---is not at all obvious:
For example, even though Heron's formula \eqref{eq:Heron} is permutation-invariant, it is not actually in the form \eqref{eq:DeepSets}, if the triangle is seen as being defined by its corners (or even its sides).

A few comments about this Deep-Sets form \eqref{eq:DeepSets} we'd make are the following:
\begin{itemize}
    \item The function $\phi$ is universal to all the points $x_i$; it's form won't depend on any properties of each point, nor any properties of the overall collection of points (the \emph{point cloud}, as it were).
    \item The point properties $x_i$ and the output of the function $\phi$ might both be very large.
    That is, each point $x_i$ might be a vector or even an ordered list of vectors and tensors and images and so on, and the output of the function $\phi$ might also be very large and complicated.
    The output of the function $\phi$ is some kind of informative, latent encoding of the point $x_i$.
    \item If a function can be written in Deep-Sets form \eqref{eq:DeepSets}, then that function can be executed in a map--reduce framework, in which the function $\phi$ is executed independently and asynchronously on each data point (maybe even in the data center) and the sums are performed hierarchically or in any order.
    That means that for large problems executed on large data sets in large data centers, problems in this form \eqref{eq:DeepSets} can be performed in wall-clock time that scales as $\log N$.
\end{itemize}



\section{A note on our method}

\section{Result and discussion}

\begin{align}
A^2 = &
-\frac{ 54}{368}\,(\sum_{i=1}^3 x_i^2\,y_i^2) 
-\frac{258}{368}\,(\sum_{i=1}^3 x_i\,y_i)^2 \nonumber \\ &
+\frac{285}{368}\,(\sum_{i=1}^3 x_i^2)\,(\sum_{i=1}^3 y_i^2) 
+\frac{ 36}{368}\,(\sum_{i=1}^3 y_i)\,(\sum_{i=1}^3 x_i^2\,y_i) \nonumber \\ &
+\frac{ 36}{368}\,(\sum_{i=1}^3 x_i)\,(\sum_{i=1}^3 x_i\,y_i^2) 
-\frac{101}{368}\,(\sum_{i=1}^3 y_i)^2\,(\sum_{i=1}^3 x_i^2) \nonumber \\ &
+\frac{148}{368}\,(\sum_{i=1}^3 x_i)\,(\sum_{i=1}^3 y_i)\,(\sum_{i=1}^3 x_i\,y_i) 
-\frac{101}{368}\,(\sum_{i=1}^3 x_i)^2\,(\sum_{i=1}^3 y_i^2) \nonumber \\ &
+\frac{  9}{368}\,(\sum_{i=1}^3 x_i)^2\,(\sum_{i=1}^3 y_i)^2 
\end{align}

\end{document}
