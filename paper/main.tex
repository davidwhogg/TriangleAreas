% Copyright 2025 the authors.

% to-do items
% -----------
% - CH: Include the d=3, d=4 solutions and comments on d>4.
% - CH: Rewrite to remove the Hogg solution.
% - CH: Show that there is NOT an exact solution at polynomial degree 2 order.
% - CH: Needs a clear tikz figure showing the corners and notation. Like an infographic.
% - HOGG: Discuss alternatives like Novara's; And alternatives like Kaze's. Positional encodings! CH: Describe the connection to positional encodings (in, eg, transformers). Oh and cosmology and power spectrum and bispectrum.

\documentclass[12pt]{article}
\usepackage[letterpaper]{geometry}
\usepackage[hidelinks]{hyperref}
\usepackage{amsmath}

% typesetting issues
\newcommand{\documentname}{\textsl{Note}}
\newcommand{\sectionname}{Section}
\newcommand{\secref}[1]{\sectionname~\ref{#1}}
\setlength{\textwidth}{5.50in}
\setlength{\oddsidemargin}{3.25in}\addtolength{\oddsidemargin}{-0.5\textwidth}
\addtolength{\topmargin}{-0.20in}
\addtolength{\textheight}{1.40in}
\sloppy\sloppypar\raggedbottom\frenchspacing
\pagestyle{myheadings}
\markboth{foo}{\sffamily Hainje, Hogg, \& Villar / A new formula for the area of a triangle}

\usepackage{natbib}
\usepackage{aas_macros}
\bibliographystyle{aasjournal}
\setcitestyle{aysep={}}
\usepackage{setspace}
\setlength{\bibsep}{0pt plus 0.3ex}

\begin{document}\thispagestyle{empty}

\section*{\raggedright
A new formula for the area of a triangle:
Useless, but explicitly in Deep Sets and map--reduce form\footnote{%
The authors thank Alexander Novara (NYU), Ben Wandelt (Sorbonne), and Kaze Wong (Flatiron) for valuable discussions.}}

\noindent \textbf{Connor Hainje},\footnote{Email at \href{mailto:connor.hainje@nyu.edu}{connor.hainje@nyu.edu}. CH is at the Center for Cosmology and Particle Physics, Department of Physics, New York University.}
\noindent \textbf{David W. Hogg},\footnote{DWH is at the Center for Cosmology and Particle Physics, Department of Physics, New York University; the Max-Planck-Insitut f\"ur Astronomie; and the Center for Computational Astrophysics, Flatiron Institute. The Flatiron Institute is a division of the Simons Foundation.} and
\textbf{Soledad Villar}\footnote{SV is at the Department of Applied Mathematics and Statistics, Johns Hopkins University; and the Mathematical Institute for Data Science, Johns Hopkins University.}

\paragraph{Abstract:}
It is known that any permutation-invariant function of data points $r_i$
can be written in the form $\rho(\sum_i\phi(r_i))$,
where $\rho$ and $\phi$ are (probably nonlinear) functions, and the function $\phi$ might output a high-dimensional latent encoding of each point $r_i$.
This form---known in the machine-learning literature as Deep Sets---is a map--reduce form, because the $\phi$ executions can be performed asynchronously and the sums can be performed hierarchically or in any order.
The area of a triangle is a permutation-invariant function of the locations $r_i$ of the three corners $1\leq i\leq 3$.
We find, possibly for the first time, the formula for the area of a triangle, given the positions of the three corners, that is explicitly in Deep Sets form.
% It is the square-root of a permutation-invariant quartic polynomial involving the integer 368.
No insights of any kind were gained from this work.

\section{Introduction}
The area $A$ of a triangle can be written in many ways.
Here are a few:
\begin{align}
A &= \frac{1}{2}\,\text{(base)}\,\text{(height)} \label{eq:school} \\
  &= \frac{1}{2}\,|(\vec{r}_3 - \vec{r}_1) \times (\vec{r}_2 - \vec{r}_1)| \label{eq:cross} \\
  &= \frac{1}{2}\,a\,b\,\sin(C) \label{eq:sine} \\
  &= \sqrt{s\,(s-a)\,(s-b)\,(s-c)} \label{eq:Heron} ~,
\end{align}
where the first formula \eqref{eq:school} is the usual introduction to the subject,
$\vec{r}_1, \vec{r}_2, \vec{r}_3$ are three vectors pointing from an origin to the corners of the triangle,
$\times$ represents the cross-product operator (or its generalization to higher dimensions),
$a, b, c$ are side lengths of the triangle,
$C$ is the angle between the $a$ and $b$ sides,
and $s$ is the semi-perimeter $(a + b + c)/2$.
Below, we present a new formula for the area of a triangle, which we believe may be novel.

While the area of a triangle does not depend on how its sides or corners are labeled,
only the last of these formulae---Heron's formula \eqref{eq:Heron}---is explicitly permutation-invariant.

The theory of permutation-invariant functions says that any per\-mu\-tation-invariant function (under conditions??) of $N$ objects $r_i$ (with $1\leq i\leq N$) can be approximated (to arbitrary precision??) with an expression of the form \citep{zaheer+17-Deep-Sets}
\begin{equation}
    \rho(\sum_{i=1}^N\phi(r_i)) \label{eq:DeepSets}
\end{equation}
where $\rho$ and $\phi$ are (possibly very nonlinear) functions,
and the $r_i$ are the data points that can be permuted without changing the result.
This result---which we associate with the subject \emph{Deep Sets} (CITE)---is not at all obvious:
For example, even though Heron's formula \eqref{eq:Heron} is permutation-invariant, it is not actually in the form \eqref{eq:DeepSets}, if the triangle is seen as being defined by its corners (or even its sides).

A few comments about this Deep Sets form \eqref{eq:DeepSets} we'd make are the following:
\begin{itemize}
    \item The function $\phi$ is universal to all the points $r_i$; its form won't depend on label $i$ or on any properties of each point, nor any properties of the overall collection of points (the \emph{point cloud}, as it were).
    \item The point properties $r_i$ and the output of the function $\phi$ might both be very large.
    That is, each point $r_i$ might be a vector or even an ordered list of vectors and tensors and images and so on, and the output of the function $\phi$ might also be very large and complicated.
    The output of the function $\phi$ is some kind of informative, latent encoding of the point $r_i$.
    \item If a function can be written in Deep-Sets form \eqref{eq:DeepSets}, then that function can be executed in a map--reduce framework, in which the function $\phi$ is executed independently and asynchronously on each data point (maybe even in the data center) and the sums are performed hierarchically or in any order.
    That means that for large problems executed on large data sets in large data centers, problems in this form \eqref{eq:DeepSets} can often be performed in wall-clock time that scales as $\log N$.
\end{itemize}

Although the contribution of this paper is minimal, or even zero, it was inspired by real problems in physics.
For one very specific example, all known exact formulae for the gravitational potential energy of a system of $N$ point masses involve sums over pairs of points (and also require taking differences of points); they have not ever been written in Deep Sets form \eqref{eq:DeepSets} to our knowledge.
CH: THAT SAID, THERE ARE MULTIPOLE-LIKE EXPRESSIONS THAT MIGHT BE IN DEEPSETS AND APPROXIMATE THIS CALCULATION; SEE THE DISCUSSION?
In cosmology, standard estimators for $k$-point functions (correlation functions, bispectra, and so on) involve sums over $k$-tuples; once again they have not been written in the Deep Sets form.
CH: DO THE PHILCOX PAPERS ADDRESS THIS PREVIOUS POINT?
Importantly, when these statistics are computed in their naive forms, they are very time-consuming because they involve sums over all elements of $N$-choose-$k$, which scales (badly) as $N^k$.
Because 3-point functions involve the shapes and areas of triangles, consideration of the 3-point function led to the question answered in this \documentname:
\emph{What is the Deep Sets (or map--reduce) expression for the area of a triangle?}
We deliver in \secref{sec:result}.

\section{A note on our method}\label{sec:method}
How did we find the formula \eqref{eq:result} presented below in \secref{sec:result}?
Fundamentally, our method took the form of a symbolic regression, followed by a confirmation with computational algebra.

For the symbolic regression, we considered a linear combination of monomials, each of which can be written in Deep Sets form, but each of which has units of area-squared (position to the fourth power).
We fit a linear combination of the monomials to a data set of generated triples of points, drawn from a two-dimensional Gaussian whose mean was also drawn from a Gaussian in a two-dimensional space.
For each triple we attach a label, which is the squared area, and we require the linear combination of fourth-order, Deep Sets monomials to reproduce the labels to machine precision.
The simple code is available for re-use under an open-source license at \url{https://github.com/davidwhogg/TriangleAreas}.

We choose to work with squared area $A^2$ and not area $A$ because, geometrically, areas are signed quantities.
Indeed, the formulae \eqref{eq:school} through \eqref{eq:Heron} either (implicitly) involve square roots of polynomials, or else absolute value signs, such that the corresponding pure polynomial expression would be for the squared area.
We restricted to monomials that contain four powers of positional coordinates because any expression for the squared area must be units covariant (equivariant to changes of the units used for the measurements of positions or lengths).

Perhaps non-obviously, for a monomial to be in Deep Sets form, it must consist only of products of sums over the three points.
It cannot contain any differences, or any sums over pairs of points (or edges).
Thus $(\sum_{i=1}^2 x_i^2)\,(\sum_{i=1}^3 x_i\,y_i)$ is a permitted monomial, but $(\sum_{i=1}^3\sum_{j=1}^3 x_i^2\,y_j^2)$ is not a permitted monomial.
These monomials are both correct from a units perspective, but the second is not in Deep Sets form, because it contains products of position coordinates from different points.

In order to generate all the possible invariant monomials, we first note that they are made of products of factors of the form $F_{m,n} = \sum_i x_i^m y_i^n$.
The invariant monomials of degree $d$ are then all products $\prod_i F_{m_i, n_i}$ such that $\sum_i m_i + n_i = d$.

% HOGG: How to construct all the invariant polynomials.

% HAINJE: Given data of form $(x, y)$, invariant factors are $F_{m,n} = \sum_i x_i^m y_i^n$, and the invariant monomials are any products of these factors $\prod_{i} F_{m_i,n_i}$. Do we want/need to be more specific than that?

In detail, the formula for the area (or area squared) of a triangle must be invariant to permutations, translations, rotations, and reflections.
The Deep Sets form enforces only permutation-invariance.
Some of the monomials are invariant to reflections, but no individual monomial is invariant to translations or rotations.
It is left as an exercise to the reader to show that our solution \eqref{eq:result} below is invariant to translations, rotations, and reflections.

The symbolic regression produces a linear combination of monomials with real-valued coefficients.
Trial-and-error delivered the integer 368, which makes all of the coefficients obviously rational.
We have no theory of this number; we leave it as a second exercise to the reader to explain this large, non-prime integer.

Finally, the expression found is output by the code in LaTeX form, to create the typesetting commands for \eqref{eq:result} and also output by the code in Mathematica form, to permit computational algebra operations.
A Mathematica \texttt{simplify} operation [HOGG NO NOT REALLY] is used to show that the Deep Sets expression \eqref{eq:result} is identical to the classic formulae for a triangle's squared area $A^2$.

\section{Result and discussion}\label{sec:result}
Our result---and only contribution---is that the area $A$ of a triangle is given by
\begin{align}
A^2 = &
-\frac{ 54}{368}\,(\sum_{i=1}^3 x_i^2\,y_i^2) 
-\frac{258}{368}\,(\sum_{i=1}^3 x_i\,y_i)^2 \nonumber \\ &
+\frac{285}{368}\,(\sum_{i=1}^3 x_i^2)\,(\sum_{i=1}^3 y_i^2) 
+\frac{ 36}{368}\,(\sum_{i=1}^3 y_i)\,(\sum_{i=1}^3 x_i^2\,y_i) \nonumber \\ &
+\frac{ 36}{368}\,(\sum_{i=1}^3 x_i)\,(\sum_{i=1}^3 x_i\,y_i^2) 
-\frac{101}{368}\,(\sum_{i=1}^3 y_i)^2\,(\sum_{i=1}^3 x_i^2) \nonumber \\ &
+\frac{148}{368}\,(\sum_{i=1}^3 x_i)\,(\sum_{i=1}^3 y_i)\,(\sum_{i=1}^3 x_i\,y_i) 
-\frac{101}{368}\,(\sum_{i=1}^3 x_i)^2\,(\sum_{i=1}^3 y_i^2) \nonumber \\ &
+\frac{  9}{368}\,(\sum_{i=1}^3 x_i)^2\,(\sum_{i=1}^3 y_i)^2 ~, \label{eq:result}
\end{align}
where the two-dimensional position of each of the three corners $i$ is $r_i \equiv (x_i, y_i)$.
This answer is in Deep Sets form:
Every expression in parentheses in \eqref{eq:result} is a pure sum over all three points.
There are no sums over differences, and there is no ordering of the points.
Being very explicit, this can be written in the Deep Sets form \eqref{eq:DeepSets} if we define $\rho$ and $\phi$ as follows:

HOGG: Explicit vector $\phi()$ and explicit expression $\rho()$.

HAINJE: Here it is:
\begin{gather}
    \phi(r_i) = (x_i, y_i, x_i\,y_i, x_i^2, y_i^2, x_i\,y_i^2, x_i^2\,y_i, x_i^2\,y_i^2) \\
    \begin{aligned}
    \rho(\textstyle\sum_{i=1}^{3}\phi(r_i)\equiv \vec{u}) = &
    - \tfrac{54}{368} u_8
    - \tfrac{258}{368} u_3^2
    + \tfrac{285}{368} u_4\,u_5
    + \tfrac{36}{368} u_2\,u_7
    + \tfrac{36}{368} u_1\,u_6 \\ &
    - \tfrac{101}{368} u_2^2\,u_4
    + \tfrac{148}{368} u_1\,u_2\,u_3
    - \tfrac{101}{368} u_1^2\,u_5
    + \tfrac{9}{368} u_1^2\,u_2^2.
    \end{aligned}
\end{gather}

Note the symmetries....! See the $36/368$ terms, or the $-101/368$ terms

But note that it isn't obviously invariant to arbitrary rotations.
And it isn't obviously invariant to translations.
Formulae \eqref{eq:cross}, \eqref{eq:sine}, and \eqref{eq:Heron} are obviously rotation-invariant and translation-invariant.

HOGG SAY: CH: EXPAND THIS INTO AN OPEN QUESTIONS LIST:
\begin{itemize}
\item CH should answer this question: What's the generalization to $d$ dimensions? That is, what if the three points are each in d dimensions?
\item questionable: What happens if we try to do $A$ as a quadratic instead of $A^2$ as a quartic?
\item What happens if you hold the formula fixed, but use more than $N=3$ points?
The formula still returns a squared area.
That's the area of what?
\item What about areas of quadrangles or larger polygons?
\item What about volumes and hypervolumes of larger objects?
\end{itemize}

\newpage
By inspection, we have also found a simpler form of the same formula, which is given by
\begin{align}
    4 \, A^2 = & \
        2 \, (\sum_{i=1}^{3} x_i ) \, (\sum_{i=1}^{3} y_i ) \, (\sum_{i=1}^{3} x_i \, y_i )
        \nonumber \\ &
        + 3 \, (\sum_{i=1}^{3} x_i^2 ) \, (\sum_{i=1}^{3} y_i^2 )
        - 3 \, (\sum_{i=1}^{3} x_i \, y_i )^2.
        \nonumber \\ &
        - \, (\sum_{i=1}^{3} x_i )^2 \, (\sum_{i=1}^{3} y_i^2 )
        - \, (\sum_{i=1}^{3} y_i )^2 \, (\sum_{i=1}^{3} x_i^2 )
\end{align}
In explicit Deep Sets notation, this can be written
\begin{align}
    \phi(r_i) &= \begin{bmatrix}
        x_i & y_i & x_i\,y_i & x_i^2 & y_i^2 \end{bmatrix}^\top
        \equiv \vec{u}^\top, \\
    \rho(\sum_{i=1}^{3} \phi(r_i))
    &= \tfrac{2}{4} u_1\,u_2\,u_3
    + \tfrac{3}{4} u_4\,u_5
    - \tfrac{1}{4} u_1^2\,u_5
    - \tfrac{1}{4} u_2^2\,u_4
    - \tfrac{3}{4} u_3^2.
\end{align}

The two are related by
\begin{equation}
    4\,u_1\,u_6 + 4\,u_2\,u_7 - 6\,u_8 = 
    -u_1^2 u_2^2 + 4 u_1 u_2 u_3 - 2 u_3^2 + u_2^2 u_4 + u_1^2 u_5 - u_4 u_5 ;
\end{equation}
namely, plugging this into \eqref{eq:result} causes it to reduce immediately into this simpler form.

This form was not found by the symbolic regression of \secref{sec:method} because the regression makes use of \verb+lstsq+ with more parameters than data points. In this regime, the solution that minimizes the $\mathcal{L}_2$ norm of the weights (the coefficients of the invariant monomials) is returned. The $\mathcal{L}_2$ norm of the weights of our initial formula is $2127 / 1472 \approx 1.445$, whereas the $\mathcal{L}_2$ norm of the weights of this simpler formula is $3/2 = 1.5$.

HAINJE: Would lasso regression return my formula? Is there a yet simpler formula?

\newpage

Define the ``invariant factor'' in $d$ dimensions
\begin{equation}
    F_{m_1 \, m_2 \, \cdots} \equiv \sum_{1=1}^{3} \prod_{k=1}^{d} (\vec{r}_{i})_{k}^{m_k}.
\end{equation}
This is the sum over all three particles of the product of the $d$ components of each $\vec{r}_i$, each component raised to a power. For example, in 2D, $d=2$ and this expands to
\begin{equation}
    F_{\ell m} = \sum_{i=1}^{3} x_i^\ell \, y_i^m.
\end{equation}
In 3D, $d=3$ and this gives
\begin{equation}
    F_{\ell m n} = \sum_{i=1}^{3} x_i^\ell \, y_i^m \, z_i^n.
\end{equation}
Then, the area formula found above can be extended to more dimensions with the following formulae.

For reference, the 2D formula was
\begin{equation}
4 A_{\rm 2D}^2 = +3 F_{02} F_{20} -3 F_{11}^2 - F_{20} F_{01}^2 +2 F_{11} F_{10} F_{01} - F_{02} F_{10}^2.
\end{equation}
In 3D, we have
\begin{equation}
\begin{aligned}
4 A_{\rm 3D}^2 = &
+3 F_{020} F_{002} 
+3 F_{200} F_{002} 
+3 F_{020} F_{200} \\&
-3 F_{011}^2 
-3 F_{101}^2 
-3 F_{110}^2 \\&
+2 F_{011} F_{001} F_{010} 
+2 F_{100} F_{001} F_{101}
+2 F_{100} F_{010} F_{110} \\&
-F_{002} F_{010}^2 
-F_{002} F_{100}^2 
-F_{020} F_{001}^2 
-F_{020} F_{100}^2
-F_{200} F_{010}^2
-F_{200} F_{001}^2. 
\end{aligned}
\end{equation}
And in 4D, we have
\begin{equation}
\begin{aligned}
4 A_{\rm 4D}^2 = &
+3 F_{0002} F_{0020} 
+3 F_{0002} F_{0200} 
+3 F_{0002} F_{2000} 
\\&
+3 F_{0200} F_{2000} 
+3 F_{0200} F_{0020} 
+3 F_{0020} F_{2000} 
\\&
-3 F_{0011}^2
-3 F_{0101}^2
-3 F_{0110}^2
-3 F_{1001}^2 
-3 F_{1010}^2 
-3 F_{1100}^2
\\&
+2 F_{0010} F_{0011} F_{0001} 
+2 F_{0101} F_{0100} F_{0001} 
+2 F_{1001} F_{1000} F_{0001}
\\&
+2 F_{0010} F_{0100} F_{0110} 
+2 F_{0010} F_{1010} F_{1000}
+2 F_{1100} F_{0100} F_{1000} 
\\&
- F_{0020} F_{0001}^2
- F_{0200} F_{0001}^2
- F_{2000} F_{0001}^2
- F_{0002} F_{0010}^2
- F_{0200} F_{0010}^2
- F_{2000} F_{0010}^2
\\&
- F_{0002} F_{0100}^2 
- F_{0020} F_{0100}^2 
- F_{2000} F_{0100}^2
- F_{0002} F_{1000}^2 
- F_{0020} F_{1000}^2 
- F_{0200} F_{1000}^2
\end{aligned}
\end{equation}

\newpage
{\small\singlespacing\bibliography{main}}

\end{document}
