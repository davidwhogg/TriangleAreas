% Copyright 2025 the authors.

% to-do items
% -----------
% - CH: Show that there is NOT an exact solution at polynomial degree 2 order.
% - HOGG: Discuss alternatives like Novara's; And alternatives like Kaze's. Positional encodings! CH: Describe the connection to positional encodings (in, eg, transformers). Oh and cosmology and power spectrum and bispectrum.

% style notes
% -----------
% - Use \abs{} not |
% - Put variable names next to the precise thing they label.

\documentclass[12pt]{article}
\usepackage[letterpaper]{geometry}
\usepackage[hidelinks]{hyperref}
\usepackage{amsmath,amsfonts,mathtools,tikz}
\usetikzlibrary{angles,calc,quotes}

\usepackage[utf8]{inputenc}
\usepackage[T1]{fontenc}
\usepackage{lmodern}

% typesetting issues
\newcommand{\documentname}{\textsl{Note}}
\newcommand{\sectionname}{Section}
\newcommand{\secref}[1]{\sectionname~\ref{#1}}
\setlength{\textwidth}{5.50in}
\setlength{\oddsidemargin}{3.25in}\addtolength{\oddsidemargin}{-0.5\textwidth}
\addtolength{\topmargin}{-0.20in}
\addtolength{\textheight}{1.40in}
\sloppy\sloppypar\raggedbottom\frenchspacing
\pagestyle{myheadings}
\markboth{foo}{\sffamily Hainje, Hogg, \& Villar / A new formula for the area of a triangle}

\usepackage{natbib,setspace,aas_macros}
\bibliographystyle{aasjournal}
\setcitestyle{aysep={}}
\setlength{\bibsep}{0pt plus 0.3ex}

\newcommand{\abs}[1]{|\,{#1}\,|}
\renewcommand{\paragraph}[1]{\par\addvspace{1.5ex}\noindent\textsl{#1}~---}

\usepackage{xcolor}
\newcommand{\CH}[1]{{\color{blue} (CH says: #1)}}
\newcommand{\Hogg}[1]{{\color{violet} (Hogg says: #1)}}
\newcommand{\pseudosection}[1]
{}
% {\noindent{\color{green!66!black} (\textit{#1})}\par}

% define the affiliation symbols
\newcommand{\aOne}{\textsuperscript{\textasteriskcentered}}
\newcommand{\aTwo}{\textsuperscript{\textdagger}}
\newcommand{\aThree}{\textsuperscript{\textdaggerdbl}}
\newcommand{\aFour}{\textsuperscript{\textsection}}
\newcommand{\aFive}{\textsuperscript{\textbardbl}}
\newcommand{\aSix}{\textsuperscript{\textparagraph}}

\begin{document}\thispagestyle{empty}

\section*{\raggedright
A new formula for the area of a triangle:
Useless, but explicitly in Deep Sets form}

\noindent 
\textbf{Connor Hainje}\aOne\aSix,
\textbf{David W. Hogg}\aOne\aTwo\aThree, and
\textbf{Soledad Villar}\aFour\aFive

\vspace{0.5em}
{\footnotesize\par\noindent \aOne\textsl{
Center for Cosmology and Particle Physics, Department of Physics, New York University}}
{\footnotesize\par\noindent \aTwo\textsl{
Max-Planck-Insitut f\"ur Astronomie}}
{\footnotesize\par\noindent \aThree\textsl{
Center for Computational Astrophysics, Flatiron Institute}}
{\footnotesize\par\noindent \aFour\textsl{
Department of Applied Mathematics and Statistics, Johns Hopkins University}}
{\footnotesize\par\noindent \aFive\textsl{
Mathematical Institute for Data Science, Johns Hopkins University}}
{\footnotesize\par\noindent \aSix\textsl{
Email:} \texttt{\href{mailto:connor.hainje@nyu.edu}{connor.hainje@nyu.edu}}}
{\footnotesize\par}  % needed to prevent some weird spacing issue


\bigskip
\paragraph{Abstract}
It is known that any permutation-invariant function of data points $\vec{r}_i$
can be written in the form $\rho(\sum_i\phi(\vec{r}_i))$,
where $\rho$ and $\phi$ are functions.
%, and the function $\phi$ might output a high-dimensional latent encoding of each point $r_i$.
This form---known in the machine-learning literature as Deep Sets---generates a map--reduce algorithm, because the $\phi$ executions can be performed asynchronously and the sums can be performed in any order.
The area of a triangle is a permutation-invariant function of the locations $\vec{r}_i$ of the three corners $1\leq i\leq 3$.
We find, possibly for the first time, the formula for the area of a triangle that is explicitly in Deep Sets form.
This project was motivated by questions about the fundamental computational complexity of $n$-point statistics in cosmology; that said, no insights of any kind were gained from these results.

\bigskip
{\centering $\ast~\ast~\ast$ \par}
\bigskip

\pseudosection{Areas}
\noindent
The area $\Delta$ of a triangle can be written in many ways.
Here are a few:
\begin{align}
\Delta 
&= \frac{1}{2} \, \text{(base)} \, \text{(height)} \label{eq:school} \\
&= \frac{1}{2}\,a\,b\,\sin(C) \label{eq:sine} \\
&= \frac{1}{2}\, \abs{\vec{a} \times \vec{b}}
    = \frac{1}{2}\, \abs{\vec{a} \wedge \vec{b}}\label{eq:cross} \\
&= \frac{1}{2}\, \abs{
    x_1 \, y_2 - x_2 \, y_1 +
    x_2 \, y_3 - x_3 \, y_2 +
    x_3 \, y_1 - x_1 \, y_3
}\label{eq:polynomial} \\
&= \sqrt{s\,(s-a)\,(s-b)\,(s-c)} \label{eq:Heron} ~,
\end{align}
where the quantities used are defined in Figure~\ref{fig:triangle}, apart from the side lengths
$a = \abs{\vec{a}}$, $b = \abs{\vec{b}}$, $c = \abs{\vec{c}}$
and the semi-perimeter $s = (a + b + c)/2$.
Formula~\eqref{eq:school} is the usual introduction to the subject, traceable back even to Euclid.
% \eqref{eq:sine} is the side-angle-side formula.
Formula~\eqref{eq:cross} generalizes to triangles in higher dimensional spaces, using the fact that the cross or wedge product has a magnitude equal to the area of the parallelogram spanned by $\vec{a}$ and $\vec{b}$.
Formula~\eqref{eq:polynomial} gives a polynomial in terms of the Cartesian coordinates of the corners in two dimensions; it is equal to an expansion of \eqref{eq:cross}.
Formula~\eqref{eq:Heron} is Heron's formula, which elegantly computes the area from the side lengths alone.
For a collection of 110 unique triangle area formulae, see \citet{baker1885a,baker1885b}.

\begin{figure}[t!]
    \centering
    \begin{tikzpicture}
    \coordinate (F) at (-1.5,4.2);
    \coordinate (O) at (0,0);
    \coordinate (A) at (3.5,4);
    \coordinate (B) at (6,1);

    \pgfmathsetmacro{\a}{sqrt(3.5*3.5 + 4*4)}
    \pgfmathsetmacro{\b}{sqrt(6*6 + 1*1)}
    \pgfmathsetmacro{\c}{sqrt((6-3.5)*(6-3.5) + (4-1)*(4-1))}
    \pgfmathsetmacro{\s}{(\a+\b+\c)/2}
    \pgfmathsetmacro{\h}{2*sqrt(\s*(\s-\a)*(\s-\b)*(\s-\c))/\b}

    \pgfmathsetmacro{\DeltaX}{6 - 0}  % Bx - Ox
    \pgfmathsetmacro{\DeltaY}{1 - 0}  % By - Oy
    \pgfmathsetmacro{\len}{sqrt(\DeltaX*\DeltaX + \DeltaY*\DeltaY)}

    \def\bLen{0.5mm}
    \def\hLen{0.3mm}
    \pgfmathsetmacro{\bdx}{-\DeltaY/\len * \bLen}
    \pgfmathsetmacro{\bdy}{\DeltaX/\len * \bLen}
    \pgfmathsetmacro{\hdx}{\DeltaX/\len * \hLen}
    \pgfmathsetmacro{\hdy}{\DeltaY/\len * \hLen}
    \pgfmathsetmacro{\hx}{-\DeltaY/\len * \h}
    \pgfmathsetmacro{\hy}{\DeltaX/\len * \h}

    \coordinate (BaseO) at ($(O) - (\bdx,\bdy)$);
    \coordinate (BaseB) at ($(B) - (\bdx,\bdy)$);
    \coordinate (HeightO) at ($(B) + (\hdx,\hdy)$);
    \coordinate (HeightA) at ($(B) + (\hdx,\hdy) + (\hx,\hy)$);

    % mark points
    \fill (O) circle (1.2pt) node[below right]{$(x_1, y_1)$};
    \fill (A) circle (1.2pt) node[above, yshift=+1mm]{$(x_2, y_2)$};
    \fill (B) circle (1.2pt) node[below right]{$(x_3, y_3)$};
    \fill[gray] (F) circle (1.2pt) node[above left]{$(0, 0)$};

    % draw vectors
    \def\pad{1.5mm}
    \draw[->, line width=0.4mm]
        ($(O)!\pad!(A)$) -- ($(A)!\pad!(O)$)
        node[midway, left, xshift=-1mm, yshift=+1mm] {$\vec{a}$};
    \draw[->, line width=0.4mm]
        ($(O)!\pad!(B)$) -- ($(B)!\pad!(O)$)
        node[midway, below, xshift=+1mm, yshift=-0.5mm] {$\vec{b}$};
    \draw[->, line width=0.4mm]
        ($(A)!\pad!(B)$) -- ($(B)!\pad!(A)$)
        node[midway, right, xshift=+1mm, yshift=+1mm] {$\vec{c}$};

    \draw[->, line width=0.1mm, gray]
        ($(F)!\pad!(O)$) -- ($(O)!2mm!(F)$)
        node[near start, left, xshift=-1mm, yshift=0mm, black] {$\vec{r}_1$};
    \draw[->, line width=0.1mm, gray]
        ($(F)!\pad!(A)$) -- ($(A)!2mm!(F)$)
        node[near start, above, xshift=+1mm, yshift=+1mm, black] {$\vec{r}_2$};
    \draw[->, line width=0.1mm, gray]
        ($(F)!\pad!(B)$) -- ($(B)!3mm!(F)$)
        node[very near start, right, xshift=-2mm, yshift=-4mm, black] {$\vec{r}_3$};

    % draw arc for angle
    \draw pic["$C$", draw=black, angle radius=12mm, angle eccentricity=1.3] {angle=B--O--A};

    % draw base, height guide lines
    \draw[dotted, line width=0.3mm]($(O)!\pad!(BaseO)$) -- ($(BaseO)$);
    \draw[dotted, line width=0.3mm]($(B)!\pad!(BaseB)$) -- ($(BaseB)$);
    \draw[dotted, line width=0.3mm]($(B)!\pad!(HeightO)$) -- ($(HeightO)$);
    \draw[dotted, line width=0.3mm]($(A)!\pad!(HeightA)$) -- ($(HeightA)$);

    % draw base, height size lines
    \draw[<->]
        ($(BaseO)!\pad!(O)!\pad!(BaseB)$) -- ($(BaseB)!\pad!(B)!\pad!(BaseO)$)
        node[midway, below, xshift=+2mm, yshift=-1mm] {base};
    \draw[<->]
        ($(HeightO)!\pad!(O)!\pad!(HeightA)$) -- ($(HeightA)!\pad!(A)!\pad!(HeightO)$)
        node[midway, right, xshift=+1mm, yshift=+1mm] {height};
    
    \end{tikzpicture}
    \caption{A triangle.}
    \label{fig:triangle}
\end{figure}

The area of a triangle has many symmetries. It is invariant to translation, rotation, and reflection, as well as to the labeling of sides or corners. However, these symmetries are not always obvious from a given formula; for example, of those listed above, only Heron's formula \eqref{eq:Heron} is explicitly permutation-invariant. \Hogg{Isn't formula \eqref{eq:polynomial} also explicitly permutation invariant?}

% \CH{struggling to find a segue here} \Hogg{We could call out symmetries: These formulae are for a quantity that obeys permutation, translation, rotation, and reflection symmetries. But these symmetries aren't always visible. Then go to the point (above) about Heron's formula.}

\pseudosection{Deep Sets}

A function $f(\vec{r}_1, \dots, \vec{r}_N)$ is invariant to permutation of its arguments if and only if it can be decomposed in the form
\begin{equation}
    f(r_1, \dots, r_N) = \rho \, \big( \sum_{i=1}^{N} \, \phi(\vec{r}_i) \big), \label{eq:DeepSets}
\end{equation}
for some functions $\rho$ and $\phi$. This result---often known by the name of the deep learning architecture derived from it, \emph{Deep Sets} (\citealt{zaheer+17deepsets}; see Theorem~2)---is not at all obvious.
For example, even though Heron's formula \eqref{eq:Heron} is explicitly permutation-invariant, it is not actually in the form of \eqref{eq:DeepSets}. 
Further, despite the triangle area being permutation-invariant, there is no existing \Hogg{previously known?} formula for the area of triangle which is explicitly in this form.

\pseudosection{Computational motivations}

% Although the contribution of this paper is minimal, or even zero, it was inspired by real problems in machine learning and physics. \CH{another bad segue}

Although such a formula is of minimal, or even zero, practical interest, our search is motivated by real problems in machine learning and physics.
For example, in machine learning, one area of relevance is deep learning on point clouds (e.g. sets of three-dimensional position vectors) \citep{guo2020pointclouds}.
Such methods are used for problems such as three-dimensional shape classification and object detection.

\pseudosection{Cosmology motivations}

The laws of classical physics are generally invariant to permutation.
For example, the gravitational or electric potential energies of a set of point masses or point charges is invariant to the ordering or labeling of the points.
Further, and important to our own motivations, cosmological clustering statistics such as 
the correlation function \citep{},
the power spectrum \citep{},
the bispectrum \citep{},
and higher-order statistics \citep{}
used for precision measurement in cosmology \citep{}
are all permutation-invariant.
All of these statistics na\"ively involve loops over all $k$-tuples of points, which not only is inconsistent with Deep Sets form, but which has a runtime that scales as $N^k$, where $N$ the number of points in the set.

\pseudosection{Scalings}

Because all these (computationally expensive) statistics are permutation-invariant, they must be possible to write in Deep Sets form.
At the same time, Deep Sets form \eqref{eq:DeepSets} looks like it is computable in linear time. 
What resolves this discrepancy?
The resolution must be either that $\phi$ has a dimensionality that grows with $N$, or else that the complexity of $\rho$ grows with $N$.

Even if $\phi$ and $\rho$ grow rapidly with $N$, it might be that Deep Sets form \eqref{eq:DeepSets} delivers a computational advantage in that it is explicitly in map--reduce form.
Map--reduce is a model for parallelizing and distributing a calculation over a large dataset.
In our case, $\phi$ is the ``map'':
It can be executed independently and asynchronously on each data point.
The sums are then the ``reduce'', which can be performed hierarchically across a data center such that the runtime of the sums scales only logarithmically with the number of points.
(Granted, a triangle will never have more than three points!)

\pseudosection{Result}

These considerations led us to search for Deep Sets forms for problems that depend on pairs, triples, or $k$-tuples of points in a set. 
The simplest such problem is the area of a triangle, which depends on pairs of points via
side lengths, as in Heron's formula \eqref{eq:Heron};
angles, as in \eqref{eq:sine};
or point displacements, as in \eqref{eq:cross}.
In this search, we find the following Deep Sets-form formula for the area $\Delta$ of a triangle:
\begin{align}
    \Delta^2 = \ &
    \frac{3}{4} \, \big( \sum_{i=1}^{3} x_i^2 \big) \, \big( \sum_{i=1}^{3} y_i^2 \big)
    - \frac{3}{4} \, \big( \sum_{i=1}^{3} x_i \, y_i \big)^2 
    + \frac{2}{4} \, \big( \sum_{i=1}^{3} x_i \, y_i \big) \, \big( \sum_{i=1}^{3} x_i \big) \, \big( \sum_{i=1}^{3} y_i \big)
    \nonumber\\ &
    - \, \frac{1}{4} \, \big( \sum_{i=1}^{3} y_i^2 \big) \, \big( \sum_{i=1}^{3} x_i \big)^2
    - \, \frac{1}{4} \, \big( \sum_{i=1}^{3} x_i^2 \big) \, \big( \sum_{i=1}^{3} y_i \big)^2
     ~,
\label{eq:result}
\end{align}
where the two-dimensional position of corner $i$ is $\vec{r}_i \equiv (x_i, y_i)$.
This equation \eqref{eq:result} is the primary contribution of this \documentname.

We verified equation \eqref{eq:result} via computational algebra, expanding and simplifying the formula to demonstrate its equivalence to \eqref{eq:polynomial}.
% \Hogg{Perhaps we should re-write cross to be in terms of differences again? Also, Figure~1 doesn't define the $\vec{r}_i$ variables, which in principle should be shown as displacements from an origin.}
\Hogg{Try to make this text less narrative, and always stay in present tense! We are not telling a story of our lives.}
The formula is in Deep Sets form, since
every expression in parentheses in \eqref{eq:result} is a pure sum over all three points.
% There are no sums over differences, and there is no ordering of the points.
Being very explicit, it matches \eqref{eq:DeepSets} if we define $\rho$ and $\phi$ as follows:
\begin{gather}
    \phi(\vec{r}_i) = (
        x_i, \,
        y_i, \,
        x_i^2, \,
        x_i \, y_i, \,
        y_i^2
    ), \quad
    (
        F_{10}, \,
        F_{01}, \,
        F_{20}, \,
        F_{11}, \,
        F_{02}
    ) = \sum_i \phi(\vec{r}_i),
    \nonumber\\
    \rho \, \big( \sum_i \phi(\vec{r}_i) \big)
    = \tfrac{3}{4} \, F_{20} \, F_{02}
    - \tfrac{3}{4} \, F_{11}^2
    + \tfrac{2}{4} \, F_{11} \, F_{10} \, F_{01}
    - \tfrac{1}{4} \, F_{02} \, F_{10}^2
    - \tfrac{1}{4} \, F_{20} \, F_{01}^2.
\end{gather}

\pseudosection{A note on symbolic regression}

Note that a triangle defined by three points has one-half the area of the parallelogram spanned by the vectors $\vec{a} = \vec{r}_2 - \vec{r}_1$ and $\vec{b} = \vec{r}_3 - \vec{r}_1$.
Multiplying both sides by 4 reveals that \eqref{eq:result} is an integer polynomial of invariant factors for the squared area of the parallelogram.
Using this fact, we have written a program to perform a symbolic regression which recovers our result.\footnote{
    The simple code is available for re-use under an open-source license at \url{https://github.com/davidwhogg/TriangleAreas}.
}

In this symbolic regression, we considered a linear combination of monomials, each of which can be written in Deep Sets form, but each of which has units of area-squared (position to the fourth power).
All the possible monomials are made of permutation-invariant factors of the form 
\begin{equation}
F_{mn} = \sum_{i=1}^{3} x_i^m \, y_i^n.
\end{equation}
Such a factor has degree $m + n$.
The invariant monomials of degree $d$ are then products $\prod_j F_{m_j n_j}$ such that $\sum_j m_j + n_j = d$.
We generated all factors $F_{mn}$ with degree $1 \leq m + n \leq 4$, then formed all products of these factors with total degree four.
However, these monomials are not all linearly-independent;
the five monomials made of a single degree-four factor ($F_{40}$, $F_{31}$, $F_{22}$, $F_{13}$, $F_{04}$) can be written as linear combinations of the others and are thus removed.
This leaves \CH{number?} monomials.

We fit a linear combination of the monomials to a data set of generated triples of points, drawn from a two-dimensional Gaussian whose mean was also drawn from a Gaussian in a two-dimensional space.
Each triple defines a parallelogram whose squared area we use as a label for regression.
We then perform least-squares regression to fit a linear combination of the invariant monomials.
We require that the labels are reproduced to machine precision after rounding the coefficients to the nearest integers.

\pseudosection{Generalizations: higher dimensions}

If the triangle is embedded in a higher-dimensional space with the positions of the three vertices given by $\vec{r}_i \in \mathbb{R}^{d}$ for $d \geq 2$, the area of the triangle can be determined by summing in quadrature the areas of the triangles made by all projections to two axes. This is easiest to see starting from \eqref{eq:cross}, using the wedge product for vectors $\vec{a}, \vec{b}$ in a higher-dimensional space, as
\begin{align}
    \Delta^2
    &= \frac{1}{4}\, \abs{\vec{a} \wedge \vec{b}}^2 \nonumber\\
    &= \frac{1}{4}\, \abs{\vec{a}}^2 \, \abs{\vec{b}}^2 - \frac{1}{4}\, \abs{\vec{a} \cdot \vec{b}}^2 \nonumber\\
    &= \frac{1}{4} \sum_{1 \leq j, k \leq d} ( a_j \, a_j \, b_k \, b_k - a_j \, b_j \, a_k \, b_k ) \nonumber\\
    &= \frac{1}{4} \sum_{1 \leq j < k \leq d} ( a_j \, b_k - a_k \, b_j)^2 \nonumber\\
    &= \sum_{1 \leq j < k \leq d} \Delta_{jk}^2,
\end{align}
where in the last line we have introduced $\Delta_{jk}$, the area of the triangle formed by just the $j^{\rm th}$ and $k^{\rm th}$ components of the position vectors.
% For example, in three dimensions, the area of the triangle is given by
% \begin{equation}
%     \Delta_{\rm 3D}^2 =
%     \Delta_{12}^2 + \Delta_{13}^2 + \Delta_{23}^2,
% \end{equation}
% and, in four dimensions, it is given by
% \begin{equation}
%     \Delta_{\rm 4D}^2 =
%     \Delta_{12}^2 + \Delta_{13}^2 + \Delta_{14}^2
%     + \Delta_{23}^2 + A_{24}^2
%     + \Delta_{34}^2,
% \end{equation}
By using \eqref{eq:result} for each of the two-dimensional sub-areas, each can be computed in a permutation-invariant way and the entire result is thus permutation invariant and in Deep Sets form. 
% \Hogg{This last result somehow is related to the meaning of the square of the magnitude of a bivector.}

\pseudosection{Generalizations: simplex volume}

Our method also extends to the volume of a simplex. In three dimensions, four vectors $\{ \vec{r}_1, \vec{r}_2, \vec{r}_3, \vec{r}_4 \}$ determine the vertices of a simplex. 
Like the area of a triangle is one-half the area of a related parallelogram, the volume of a simplex is one-sixth the volume of the parallelepiped spanned by the vectors $\{ \vec{r}_2 - \vec{r}_1, \vec{r}_3 - \vec{r}_1, \vec{r}_4 - \vec{r}_1 \}$. 

Expecting an integer solution once again, we modified our symbolic regression to generate 4-tuples of points, using the squared volumes of the parallelepipeds they produce as labels. 
The invariant factors then have three indices, $F_{\ell m n} = \sum_i x_i^\ell \, y_i^m \, z_i^n$, and we form all invariant monomials of degree six.
The result, which we have verified by computational algebra, is that the volume of the simplex $V$ is \Hogg{Since it is the volume that is $V$ not the simplex, it should say ``The volume $V$ of the simplex...''} given by
\begin{align}
(6 V)^2 = &
+8 \, F_{110} \, F_{101} \, F_{011}
+4 \, F_{200} \, F_{020} \, F_{002}
\nonumber \\ &
-4 \, F_{200} \, F_{011}^2
-4 \, F_{020} \, F_{101}^2
-4 \, F_{002} \, F_{110}^2
\nonumber \\ &
+2 \, F_{200} \, F_{011} \, F_{010} \, F_{001}
+2 \, F_{020} \, F_{101} \, F_{100} \, F_{001}
+2 \, F_{002} \, F_{110} \, F_{100} \, F_{010}
\nonumber \\ &
-2 \, F_{110} \, F_{101} \, F_{010} \, F_{001}
-2 \, F_{110} \, F_{011} \, F_{100} \, F_{001}
-2 \, F_{101} \, F_{011} \, F_{100} \, F_{010}
\nonumber \\ &
+1 \, F_{110}^2 \, F_{001}^2
+1 \, F_{101}^2 \, F_{010}^2
+1 \, F_{011}^2 \, F_{100}^2
\nonumber \\ &
-1 \, F_{200} \, F_{020} \, F_{001}^2
-1 \, F_{200} \, F_{002} \, F_{010}^2
-1 \, F_{020} \, F_{002} \, F_{100}^2 \, .
\label{eq:simplex}
\end{align}
There are many similarities between this and our result for the triangle area \eqref{eq:result}, which we believe may indicate a way to generalize this result to the hypervolume of a $d$-simplex.

\pseudosection{Generalizations: Heron symmetrized}

We stated above that Heron's formula \eqref{eq:Heron} is explicitly permutation-invariant, and yet it is not represented in Deep Sets form. Consider instead
\begin{equation}
\label{eq:HeronDeepSets}
    \Delta = \frac{1}{4} \, \sqrt{
        \left( a^2 + b^2 + c^2 \right)^2
        - 2 \left( a^4 + b^4 + c^4 \right)
    }~.
\end{equation}
It is simple to show that this is exactly equivalent to Heron's formula when expanded, but it is now in Deep Sets form.\footnote{
    This is not novel. Heron's formula is a special case of Brahmagupta's formula, which gives the area $K$ of a convex cyclic quadrilateral in terms of its four side lengths $a, b, c, d$ \citep{coolidge1939}.
    On Wikipedia, it has been written
    \begin{equation}
        K = \frac{1}{4} \, \sqrt{
            \left( a^2 + b^2 + c^2 + d^2 \right)^2
            + 8 \, a\,b\,c\,d
            - 2 \left( a^4 + b^4 + c^4 + d^4 \right)
        }~,
    \end{equation}
    which is exactly \eqref{eq:HeronDeepSets} if $d$ is zero.
    This expression is in Deep Sets form, with the product $a \, b \, c \, d$ 
    representable as $\exp( \sum_{x \in \{ a,b,c,d \}} \log x )$.
    % taken to be the exponential of the sum of logarithms.
}

\pseudosection{Discussion: other symmetries}

The area of a triangle is not only invariant to permutation, but also to translation, rotation, and reflection.
Formulae \eqref{eq:sine} and \eqref{eq:Heron}, by virtue of referencing only side-lengths and angles, are invariant to these other symmetries.
While true, it is not obvious that our solution \eqref{eq:result} is invariant. An explicit proof of this is left to the reader.

\pseudosection{Discussion: sines and cosines}

\CH{add discussion on sines and cosines}

\pseudosection{Discussion: open questions}

% HOGG SAY: CH: EXPAND THIS INTO AN OPEN QUESTIONS LIST:
% \begin{itemize}
% \item questionable: What happens if we try to do $A$ as a quadratic instead of $A^2$ as a quartic?
% \item What happens if you hold the formula fixed, but use more than $N=3$ points?
% The formula still returns a squared area.
% That's the area of what?
% \item What about areas of quadrangles or larger polygons?
% \item What about volumes and hypervolumes of larger objects?
% \end{itemize}

We leave the reader with the following open questions:
\begin{itemize}
    \item It is trivial to take the form of \eqref{eq:result} and evaluate the sums for more than three vectors. Does the result have any meaning?

    \item Our triangle area \eqref{eq:result} and simplex volume \eqref{eq:simplex} formulae have many similarities. We take this to imply that a similar formula should exist for the hypervolume of a $d$-simplex. What is this result?

    \item In both the triangle area and simplex volume formulae, the only invariant factors used are of degree $2$, and the factors never appear with exponents larger than $2$. Does this generalize? Can it be shown that higher order factors aren't needed?
\end{itemize}

\CH{conclude}

The authors thank Alexander Novara (NYU), Ben Wandelt (JHU), Kaze Wong (JHU), and the Blanton--Hogg group meeting at NYU for valuable discussions. The authors would also like to note that, while we believe that this result is novel, it is difficult to be sure, given the impossibility of conducting a complete literature review for such an ancient topic.

\bigskip
{\centering $\ast~\ast~\ast$ \par}

\renewcommand{\section}[2]{}%
{\small\singlespacing\bibliography{main}}

\end{document}
