% Copyright 2025 the authors.

% to-do items
% -----------
% - CH: Show that there is NOT an exact solution at polynomial degree 2 order.
% - HOGG: Discuss alternatives like Novara's; And alternatives like Kaze's. Positional encodings! CH: Describe the connection to positional encodings (in, eg, transformers). Oh and cosmology and power spectrum and bispectrum.

% style notes
% -----------
% - Use \abs{} not |
% - Put variable names next to the precise thing they label.
% - Put ~, or ~. at the end of equations that need punctuation at the end.
% - Be careful about "not translationally invariant" -- we mean "not explicitly translationally invariant" and so on. Explicitness is what this paper is about.

\documentclass[10pt]{article}
\usepackage[letterpaper]{geometry}
\usepackage[hidelinks]{hyperref}
\usepackage[utf8]{inputenc}
\usepackage[T1]{fontenc}
\usepackage{amsmath,amsfonts,lmodern,mathtools,nameref,setspace,tikz,titlesec}
\usetikzlibrary{angles,calc,quotes}

% page typesetting issues
\newcommand{\documentname}{\textsl{Note}}
\setlength{\textwidth}{4.75in}
\setlength{\oddsidemargin}{3.25in}\addtolength{\oddsidemargin}{-0.5\textwidth}
\addtolength{\topmargin}{-0.20in}
\addtolength{\textheight}{1.40in}
\sloppy\sloppypar\raggedbottom\frenchspacing
\setstretch{1.08}

% document formatting and sectioning issues
\pagestyle{myheadings}
\renewcommand{\paragraph}[1]{\par\addvspace{1.5ex}\noindent\textsl{#1}~---}
\newcommand{\secbreak}{\bigskip{\centering\footnotesize $\triangle~~~\triangle~~~\triangle$\par}\bigskip\noindent}
\titleformat*{\subsection}{\normalfont\normalsize\slshape}
\titlespacing*{\subsection}{0pt}{0pt}{0pt}
\markboth{foo}{\scshape hainje {\footnotesize\&} hogg / useless formula for the area of a triangle}
% \markboth{foo}{\scshape hainje et al. / useless formula for the area of a triangle}

% bibliography issues
\usepackage{natbib}
\bibliographystyle{aasjournal}
\setcitestyle{aysep={}}
\setlength{\bibsep}{0pt plus 0.3ex}

% math issues
\renewcommand{\d}{\mathrm{d}}
\newcommand{\abs}[1]{|\,{#1}\,|}

% issues
\usepackage{xcolor}
\newcommand{\CH}[1]{{\color{blue} (CH says: #1)}}
\newcommand{\Hogg}[1]{{\color{violet} (Hogg says: #1)}}

% affiliation symbols
\newcommand{\aOne}{\textsuperscript{\textasteriskcentered}}
\newcommand{\aTwo}{\textsuperscript{\textdagger}}
\newcommand{\aThree}{\textsuperscript{\textdaggerdbl}}
\newcommand{\aFour}{\textsuperscript{\textsection}}
\newcommand{\aFive}{\textsuperscript{\textbardbl}}
\newcommand{\aSix}{\textsuperscript{\textparagraph}}

\begin{document}\thispagestyle{empty}\setcounter{secnumdepth}{0}

\section*{\centering\normalsize\uppercase{
A formula for the area of a triangle:\\
Useless, but explicitly in Deep Sets form}}

\medskip
\noindent
\textbf{Connor Hainje}\aOne\aSix{}
\textsl{and}
\textbf{David W. Hogg}\aOne\aTwo\aThree%, and
%\textbf{Soledad Villar}\aFour\aFive

\medskip
{\footnotesize\par\noindent \aOne\textsl{
Center for Cosmology and Particle Physics, Department of Physics, New York University}}
{\footnotesize\par\noindent \aTwo\textsl{
Max-Planck-Insitut f\"ur Astronomie}}
{\footnotesize\par\noindent \aThree\textsl{
Center for Computational Astrophysics, Flatiron Institute}}
%{\footnotesize\par\noindent \aFour\textsl{
%Department of Applied Mathematics and Statistics, Johns Hopkins University}}
%{\footnotesize\par\noindent \aFive\textsl{
%Mathematical Institute for Data Science, Johns Hopkins University}}
{\footnotesize\par\noindent \aSix\textsl{
Email:} \texttt{\href{mailto:connor.hainje@nyu.edu}{connor.hainje@nyu.edu}}}
{\footnotesize\par}  % needed to prevent some weird spacing issue

\medskip
\paragraph{Abstract}
It is known that any permutation-invariant function of data points $\vec{r}_i$
can be written in the form $\rho(\sum_i\phi(\vec{r}_i))$,
where $\rho$ and $\phi$ are functions.
This form---known in the machine-learning literature as Deep Sets---also directly generates a map--reduce algorithm.
The area of a triangle is a permutation-invariant function of the locations $\vec{r}_i$ of the three corners $1\leq i\leq 3$.
We find the formula for the area of a triangle that is explicitly in Deep Sets form.
This project was motivated by questions about the fundamental computational complexity of $n$-point statistics in cosmology; that said, no insights of any kind were gained from these results.

\secbreak
The area $\Delta$ of a triangle can be written in many ways.
Here are a few:
\begin{align}
\Delta 
&= \frac{1}{2} \, (\text{base}) \, (\text{height}) \label{eq:school} \\
&= \frac{1}{2}\,a\,b\,\sin(C) \label{eq:sine} \\
&= \frac{1}{2}\, \abs{\vec{a} \times \vec{b}}
    = \frac{1}{2}\, \abs{\vec{a} \wedge \vec{b}}\label{eq:cross} \\
&= \frac{1}{2}\, \abs{
    x_1 \, y_2 - x_2 \, y_1 +
    x_2 \, y_3 - x_3 \, y_2 +
    x_3 \, y_1 - x_1 \, y_3
}\label{eq:polynomial} \\
&= \sqrt{s\,(s-a)\,(s-b)\,(s-c)} \label{eq:Heron} ~,
\end{align}
where the quantities used are defined in Figure~\ref{fig:triangle}, apart from the side lengths
$a = \abs{\vec{a}}$, $b = \abs{\vec{b}}$, $c = \abs{\vec{c}}$
and the semi-perimeter $s = (a + b + c)/2$.
Formula~\eqref{eq:school} is the usual introduction to the subject, traceable back to \citet{Euclid300BC}.
Formula~\eqref{eq:sine} makes use of the fact that the height can be found from $\sin(C) = (\text{height}) / a$, commonly seen in introductory geometry courses.
Formula~\eqref{eq:cross} generalizes to triangles in higher dimensional spaces, using the fact that the cross or wedge product has a magnitude equal to the area of the parallelogram spanned by $\vec{a}$ and $\vec{b}$.
Formula~\eqref{eq:polynomial} gives a polynomial in terms of the Cartesian coordinates of the corners in two dimensions; it is equal to an expansion of \eqref{eq:cross}.
Formula~\eqref{eq:Heron} is Heron's formula, which elegantly computes the area from the side lengths alone.
For a collection of 110 unique triangle area formulae, see \citet{Baker1885a,Baker1885b}.

\begin{figure}[t!]
    \centering
    \begin{tikzpicture}
    \coordinate (F) at (-1.5,4.4);
    \coordinate (O) at (0,0);
    \coordinate (A) at (3.5,4);
    \coordinate (B) at (6,1);

    \pgfmathsetmacro{\a}{sqrt(3.5*3.5 + 4*4)}
    \pgfmathsetmacro{\b}{sqrt(6*6 + 1*1)}
    \pgfmathsetmacro{\c}{sqrt((6-3.5)*(6-3.5) + (4-1)*(4-1))}
    \pgfmathsetmacro{\s}{(\a+\b+\c)/2}
    \pgfmathsetmacro{\h}{2*sqrt(\s*(\s-\a)*(\s-\b)*(\s-\c))/\b}

    \pgfmathsetmacro{\DeltaX}{6 - 0}  % Bx - Ox
    \pgfmathsetmacro{\DeltaY}{1 - 0}  % By - Oy
    \pgfmathsetmacro{\len}{sqrt(\DeltaX*\DeltaX + \DeltaY*\DeltaY)}

    \def\bLen{0.5mm}
    \def\hLen{0.3mm}
    \pgfmathsetmacro{\bdx}{-\DeltaY/\len * \bLen}
    \pgfmathsetmacro{\bdy}{\DeltaX/\len * \bLen}
    \pgfmathsetmacro{\hdx}{\DeltaX/\len * \hLen}
    \pgfmathsetmacro{\hdy}{\DeltaY/\len * \hLen}
    \pgfmathsetmacro{\hx}{-\DeltaY/\len * \h}
    \pgfmathsetmacro{\hy}{\DeltaX/\len * \h}

    \coordinate (BaseO) at ($(O) - (\bdx,\bdy)$);
    \coordinate (BaseB) at ($(B) - (\bdx,\bdy)$);
    \coordinate (HeightO) at ($(B) + (\hdx,\hdy)$);
    \coordinate (HeightA) at ($(B) + (\hdx,\hdy) + (\hx,\hy)$);

    % mark points
    \fill (O) circle (1.2pt) node[below right]{$(x_1, y_1)$};
    \fill (A) circle (1.2pt) node[above, yshift=+1mm]{$(x_2, y_2)$};
    \fill (B) circle (1.2pt) node[below right]{$(x_3, y_3)$};
    \fill[gray] (F) circle (1.2pt) node[above left]{$(0, 0)$};

    % draw vectors
    \def\pad{1.5mm}
    \draw[->, line width=0.4mm]
        ($(O)!\pad!(A)$) -- ($(A)!\pad!(O)$)
        node[midway, left, xshift=-1mm, yshift=+1mm] {$\vec{a}$};
    \draw[->, line width=0.4mm]
        ($(O)!\pad!(B)$) -- ($(B)!\pad!(O)$)
        node[midway, below, xshift=+1mm, yshift=-0.5mm] {$\vec{b}$};
    \draw[->, line width=0.4mm]
        ($(A)!\pad!(B)$) -- ($(B)!\pad!(A)$)
        node[midway, right, xshift=+1mm, yshift=+1mm] {$\vec{c}$};

    \draw[->, line width=0.1mm, gray]
        ($(F)!\pad!(O)$) -- ($(O)!2mm!(F)$)
        node[near start, left, xshift=-1mm, yshift=0mm, black] {$\vec{r}_1$};
    \draw[->, line width=0.1mm, gray]
        ($(F)!\pad!(A)$) -- ($(A)!2mm!(F)$)
        node[near start, above, xshift=+1mm, yshift=+1mm, black] {$\vec{r}_2$};
    \draw[->, line width=0.1mm, gray]
        ($(F)!\pad!(B)$) -- ($(B)!3.5mm!(F)$)
        node[very near start, right, xshift=-2mm, yshift=-4mm, black] {$\vec{r}_3$};

    % draw arc for angle
    \draw pic["$C$", draw=black, angle radius=12mm, angle eccentricity=1.3] {angle=B--O--A};

    % draw base, height guide lines
    \draw[dotted, line width=0.2mm]($(O)!\pad!(BaseO)$) -- ($(BaseO)$);
    \draw[dotted, line width=0.2mm]($(B)!\pad!(BaseB)$) -- ($(BaseB)$);
    \draw[dotted, line width=0.2mm]($(B)!\pad!(HeightO)$) -- ($(HeightO)$);
    \draw[dotted, line width=0.2mm]($(A)!\pad!(HeightA)$) -- ($(HeightA)$);

    % draw base, height size lines
    \draw[<->]
        ($(BaseO)!\pad!(O)!\pad!(BaseB)$) -- ($(BaseB)!\pad!(B)!\pad!(BaseO)$)
        node[midway, below, xshift=+2mm, yshift=-1mm] {base};
    \draw[<->]
        ($(HeightO)!\pad!(O)!\pad!(HeightA)$) -- ($(HeightA)!\pad!(A)!\pad!(HeightO)$)
        node[midway, right, xshift=+1mm, yshift=+1mm] {height};
    
    \end{tikzpicture}
    \caption{A triangle.}
    \label{fig:triangle}
\end{figure}

The area of a triangle has many symmetries. It is invariant to translation, rotation, and reflection, as well as to the labeling of sides or corners. However, these symmetries are not always obvious from a given formula; for example, of those listed above, only the polynomial \eqref{eq:polynomial} and Heron's formula \eqref{eq:Heron} are explicitly permutation-invariant.

A function $f(\vec{r}_1, \dots, \vec{r}_N)$ is invariant to permutation of its arguments if and only if it can be decomposed in the form
\begin{equation}
    f(r_1, \dots, r_N) = \rho \, \big( \sum_{i=1}^{N} \, \phi(\vec{r}_i) \big)~,\label{eq:DeepSets}
\end{equation}
for some functions $\rho$ and $\phi$. This result---often known by the name of the deep learning architecture derived from it, \emph{Deep Sets} (\citealt{Zaheer+17deepsets}; see Theorem~2)---is not at all obvious.
For example, even though Heron's formula \eqref{eq:Heron} is explicitly permutation-invariant, it is not actually in the form of \eqref{eq:DeepSets}. 
Further, despite the triangle area being permutation-invariant, there is no known (prior to this work) formula for the area of a triangle that is explicitly in Deep Sets form.

While this result was recently popularized as Deep Sets, it follows from the classical theory of invariant polynomials.
Specifically, it is known that the permutation-invariant polynomials on a set of any-dimensional data are representable in terms of the polarizations of the permutation-invariant polynomials on a corresponding set of one-dimensional data \citep{Weyl1939}.
Further, the permutation-invariant polynomials on a set of one-dimensional data are representable in terms of the power sums $\sum_{i=1}^{n} r_i^j$ \citep{Waring1782}.
This matches onto the Deep Sets form, as it states that a permutation-invariant polynomial of any-dimensional data $f(\{r_i\})$ can be decomposed into a polynomial ($\rho$) of polarizations of the power sums ($\sum_i \phi(\vec{r}_i)$).
\CH{this is mostly plagiarized from BBS email; need to understand it better to massage it.} \Hogg{Important to emphasize that there is a proof of universality, and it's old.}

Although a Deep Sets formula for the area of a triangle is of minimal, or even zero, practical interest, our search is motivated by real problems in physics.
The laws of classical physics are generally invariant to permutation.
For example, the gravitational (or electric) potential energy of a set of point masses (or charges) is invariant to the ordering or labeling of the points. 
Specifically, for a system of $N$ point particles with masses $m_i$ at positions $\vec{r}_i$, the gravitational potential energy is
\begin{equation}
    U = -G \, \sum_{i=1}^{N} \sum_{j=i+1}^{N} \frac{m_i \, m_j}{\abs{\vec{r}_i - \vec{r}_j}} ~.
\end{equation}
This is explicitly permutation invariant since it involves a sum without repeats over all pairs of particles, but it is not in Deep Sets form \eqref{eq:DeepSets} because it involves a double sum (with $N^2$ computational scaling) of quantities computed from pairs of particles.

Further, and important to our own motivations, cosmological clustering statistics such as 
the correlation function \citep{Peebles1973},
the power spectrum \citep{Peebles1973},
the three-point correlation function \citep{PeeblesGroth1975},
the bispectrum \citep{FrySeldner1982},
and higher-order statistics \citep{Peebles1980book}
used for precision measurement in cosmology \citep[e.g.,][]{Planck18CosmoParams,Planck18PNG,Planck18Inflation,Cabass+2022}
are all permutation-invariant.
Calculating these statistics na\"ively involves loops over all $k$-tuplets of points, which is not only inconsistent with Deep Sets form, but also has a runtime that scales as $N^k$, where $N$ the number of points in the set.

Because all these (computationally expensive) statistics are permutation-invariant, they must be possible to write in Deep Sets form.
At the same time, Deep Sets form \eqref{eq:DeepSets} looks like it is computable in linear time.
What resolves this discrepancy?
The resolution must be either that $\phi$ has a dimensionality that grows with $N$, or else that the complexity of $\rho$ grows with $N$.
Still, these time complexities may be better than the $N^k$ runtime of na\"ive implementations.

Even if $\phi$ and $\rho$ grow rapidly with $N$, it might be that Deep Sets form \eqref{eq:DeepSets} delivers a computational advantage in that it is explicitly in map--reduce form.
Map--reduce is a model for parallelizing and distributing a calculation over a large dataset \citep{DeanGhemawat2008,Lammel2008}.
In our case, $\phi$ is the ``map'':
It can be executed independently and asynchronously on each data point.
The sums are then the ``reduce'', which can be performed hierarchically across a data center such that the runtime of the sums scales only logarithmically with the number of points.
(Granted, a triangle will never have more than three points!)

So, there may be benefits to finding Deep Sets forms for problems that depend on permutation-invariant pairs, triplets, or $k$-tuplets of points.
The simplest such problem is the area of a triangle, which depends on pairs of points via
side lengths, as in Heron's formula \eqref{eq:Heron};
angles, as in \eqref{eq:sine};
or point displacements, as in \eqref{eq:cross}.

\secbreak
We find this formula---in Deep Sets form---for the area $\Delta$ of a triangle:
\begin{align}
    \Delta^2 = \ &
    \frac{3}{4} \, \big( \sum_{i=1}^{3} x_i^2 \big) \, \big( \sum_{i=1}^{3} y_i^2 \big)
    - \frac{3}{4} \, \big( \sum_{i=1}^{3} x_i \, y_i \big)^2 
    + \frac{2}{4} \, \big( \sum_{i=1}^{3} x_i \, y_i \big) \, \big( \sum_{i=1}^{3} x_i \big) \, \big( \sum_{i=1}^{3} y_i \big)
    \nonumber\\ &
    - \, \frac{1}{4} \, \big( \sum_{i=1}^{3} y_i^2 \big) \, \big( \sum_{i=1}^{3} x_i \big)^2
    - \, \frac{1}{4} \, \big( \sum_{i=1}^{3} x_i^2 \big) \, \big( \sum_{i=1}^{3} y_i \big)^2
     ~.
\label{eq:result}
\end{align}
This formula \eqref{eq:result} for the squared area\footnote{%
    We choose to work with squared area $\Delta^2$ and not area $\Delta$ because, geometrically, areas are signed quantities.
    Indeed, the formulae \eqref{eq:school} through \eqref{eq:Heron} either (implicitly) involve square roots of polynomials, or else absolute value signs, such that the corresponding pure polynomial expression would be for the squared area.}
is the primary contribution of this \documentname, along with the generalization to higher-dimensional vectors $\vec{r}_i$ below in \eqref{eq:quadrature}.

This equation \eqref{eq:result} is verified via computational algebra, expanding and simplifying the formula to demonstrate its equivalence to \eqref{eq:polynomial}.
The formula is in Deep Sets form, since
every expression in parentheses in \eqref{eq:result} is a pure sum over all three points.
Being very explicit, it matches \eqref{eq:DeepSets} for $\rho$ and $\phi$ defined as follows:
\begin{gather}
    \phi(\vec{r}_i) = (
        x_i, \,
        y_i, \,
        x_i^2, \,
        x_i \, y_i, \,
        y_i^2
    ), \quad
    (
        F_{10}, \,
        F_{01}, \,
        F_{20}, \,
        F_{11}, \,
        F_{02}
    ) = \sum_{i=1}^{3} \phi(\vec{r}_i),
    \\
    \rho \, \big( \sum_{i=1}^{3} \phi(\vec{r}_i) \big)
    = \frac{3}{4} \, F_{20} \, F_{02}
    - \frac{3}{4} \, F_{11}^2
    + \frac{2}{4} \, F_{11} \, F_{10} \, F_{01}
    - \frac{1}{4} \, F_{02} \, F_{10}^2
    - \frac{1}{4} \, F_{20} \, F_{01}^2~.
    \nonumber
\end{gather}

Note that a triangle defined by three points has one-half the area of the parallelogram spanned by the vectors $\vec{a} = \vec{r}_2 - \vec{r}_1$ and $\vec{b} = \vec{r}_3 - \vec{r}_1$.
Multiplying both sides by 4 reveals that \eqref{eq:result} is an integer polynomial of permutation-invariant factors for the squared area of the parallelogram.
Using this fact, we have written code to perform a symbolic regression that recovers our result.\footnote{
    The simple code is available for re-use under an open-source license at \url{https://github.com/davidwhogg/TriangleAreas}.
}

In this symbolic regression, we consider a linear combination of monomials, each of which can be written in Deep Sets form, but each of which has units of area-squared (position to the fourth power).
All the possible monomials are made of permutation-invariant factors of the form 
\begin{equation}
F_{mn} = \sum_{i=1}^{3} x_i^m \, y_i^n~;
\end{equation}
each such a factor has degree $m + n$.
The invariant monomials of degree $d$ are then products $\prod_j F_{m_j n_j}$ such that $\sum_j m_j + n_j = d$.
We generate all factors $F_{mn}$ with degree $1 \leq m + n \leq 4$, then form all products of these factors with total degree four.
However, these monomials are not all linearly-independent;
the five monomials made of a single degree-four factor ($F_{40}$, $F_{31}$, $F_{22}$, $F_{13}$, $F_{04}$) can be written as a linear combination of the other monomials and are thus removed.
This leaves 28 linearly-independent monomials.

We fit a linear combination of the monomials to a data set of generated triples of points, drawn from a two-dimensional Gaussian whose mean was also drawn from a Gaussian in a two-dimensional space.
Each triple defines a parallelogram whose squared area we use as a label for regression.
We then perform least-squares regression to fit a linear combination of the invariant monomials.
We require that the labels are reproduced to machine precision after rounding the coefficients to the nearest integers.

If the triangle is embedded in a higher-dimensional space with the positions of the three vertices given by $\vec{r}_i \in \mathbb{R}^{d}$ for $d \geq 2$, the area of the triangle can be determined by summing in quadrature the areas of the triangles made by all mutually orthogonal two-dimensional projections of the points:
\begin{equation}
    \Delta^2 = \sum_{1 \leq j < k \leq d} \Delta_{jk}^2~, 
    \label{eq:quadrature}
\end{equation}
where $\Delta_{jk}$ is the area of the two-dimensional triangle formed by just the $j^{\rm th}$ and $k^{\rm th}$ components of the position vectors.
A proof is given in the \nameref{sec:appendix}.
By using \eqref{eq:result} for each of the two-dimensional sub-areas, each can be computed in a permutation-invariant way and the entire result is thus explicitly permutation invariant and in Deep Sets form.

Our method also extends to the volume of a simplex. In three dimensions, four vectors $\{ \vec{r}_1, \vec{r}_2, \vec{r}_3, \vec{r}_4 \}$ determine the vertices of a simplex. 
Like the area of a triangle is one-half the area of a related parallelogram, the volume of a simplex is one-sixth the volume of the parallelepiped spanned by the vectors $\{ \vec{r}_2 - \vec{r}_1, \vec{r}_3 - \vec{r}_1, \vec{r}_4 - \vec{r}_1 \}$. 
Expecting an integer solution once again, we modify the symbolic regression to generate 4-tuplets of points, using the squared volumes of the parallelepipeds they produce as labels.
The invariant factors then have three indices, $F_{\ell m n} = \sum_i x_i^\ell \, y_i^m \, z_i^n$, and we form all invariant monomials of degree six.
The result, verifiable by computational algebra, is that the volume $V$ of the simplex is given by
\begin{align}
(6 V)^2 = &
+8 \, F_{110} \, F_{101} \, F_{011}
+4 \, F_{200} \, F_{020} \, F_{002}
\nonumber \\ &
-4 \, F_{200} \, F_{011}^2
-4 \, F_{020} \, F_{101}^2
-4 \, F_{002} \, F_{110}^2
\nonumber \\ &
+2 \, F_{200} \, F_{011} \, F_{010} \, F_{001}
+2 \, F_{020} \, F_{101} \, F_{100} \, F_{001}
+2 \, F_{002} \, F_{110} \, F_{100} \, F_{010}
\nonumber \\ &
-2 \, F_{110} \, F_{101} \, F_{010} \, F_{001}
-2 \, F_{110} \, F_{011} \, F_{100} \, F_{001}
-2 \, F_{101} \, F_{011} \, F_{100} \, F_{010}
\nonumber \\ &
+1 \, F_{110}^2 \, F_{001}^2
+1 \, F_{101}^2 \, F_{010}^2
+1 \, F_{011}^2 \, F_{100}^2
\nonumber \\ &
-1 \, F_{200} \, F_{020} \, F_{001}^2
-1 \, F_{200} \, F_{002} \, F_{010}^2
-1 \, F_{020} \, F_{002} \, F_{100}^2~.
\label{eq:simplex}
\end{align}
There are many similarities between this and our result for the triangle area \eqref{eq:result}, which we believe may indicate a way to generalize this result to the hypervolume of a $d$-simplex.

\secbreak
We now consider two more applications of the Deep Sets form to Heron's formula and to the potential energy of a system of point masses.
We stated above that Heron's formula \eqref{eq:Heron} is explicitly permutation-invariant, and yet it is not represented in Deep Sets form. Consider instead
\begin{equation}
\label{eq:HeronDeepSets}
    \Delta = \frac{1}{4} \, \sqrt{
        \left( a^2 + b^2 + c^2 \right)^2
        - 2 \left( a^4 + b^4 + c^4 \right)
    }~.
\end{equation}
It is simple to show that this is exactly equivalent to Heron's formula when expanded, but it is now in Deep Sets form.\footnote{
    This is not novel. Heron's formula is a special case of Brahmagupta's formula, which gives the area $K$ of a convex cyclic quadrilateral in terms of its four side lengths $a, b, c, d$ \citep{coolidge1939}.
    On Wikipedia, it has been written
    \begin{equation}
        K = \frac{1}{4} \, \sqrt{
            \left( a^2 + b^2 + c^2 + d^2 \right)^2
            + 8 \, a\,b\,c\,d
            - 2 \left( a^4 + b^4 + c^4 + d^4 \right)
        }~,
    \end{equation}
    which is exactly \eqref{eq:HeronDeepSets} if $d$ is zero.
    This expression is in Deep Sets form, with the product $a \, b \, c \, d$ 
    representable as $\exp( \sum_{x \in \{ a,b,c,d \}} \log x )$.
}

\CH{feels out of place...} \Hogg{This can be deleted.}
The area of a triangle is not only invariant to permutation, but also to translation, rotation, and reflection.
Formulae \eqref{eq:sine} and \eqref{eq:Heron}, by virtue of referencing only side-lengths and angles, are explicitly invariant to these other symmetries.
While true, it is not obvious that our solution \eqref{eq:result} is so invariant.
An explicit proof of this is left to the reader.

While the triangle area is related to physics problems, the polynomial methods we employ to find the Deep Sets form for the triangle area \eqref{eq:result} are not relevant in a physical context.
More relevant instead is to consider a $\phi$ which Fourier transforms the input data.
In doing so, translation symmetry is encoded as well, since the Fourier basis $\exp (i\, \vec{k}\cdot\vec{x})$ is the eigenbasis of the translation operator.
Below, we show that this can be used to produce a Deep Sets form for the gravitational potential energy of a system of point masses.

Consider a system of $N$ point particles with masses $m_i$ and positions $\vec{r}_i$.
The total energy is given in Deep Sets form by
\begin{align}
    \phi_{\vec{k}}(m_j, \vec{r}_j)
    &=m_j \, e^{-i \, \vec{k} \cdot \vec{r}_j}
    ~,
    \quad
    \sum_{i=1}^{N} \phi_{\vec{k}}(m_i, \vec{r}_i)
    = \mathcal{F}[\rho](\vec{k})~,
    \nonumber\\
    U \, \big( \sum_{i=1}^{N} \, \phi_{\vec{k}}(m_i, \vec{r}_i) \big)
    &= -\frac{G}{4 \pi^2} \,
        \int \d^3 r \,
        \mathcal{F}^{-1} \left[
            \mathcal{F}[\rho] \ast \left(
            k^{-2} \, \mathcal{F}[\rho] \right)
        \right]~.
    \label{eq:EnergyDeepSets}
\end{align}
A derivation is given in the \nameref{sec:appendix}.
In principle this requires $\phi$ to be evaluated at every possible wavenumber $\vec{k}$.
To avoid making this infinite-dimensional $\phi$, one can instead evaluate $\mathcal{F}[\rho]$ at a finite set of wavenumbers $\vec{k}$ and live with approximations to the integrals necessary for the convolution and inverse Fourier transform.
The approximation will improve as the size of $\phi$ grows.
Approaches akin to this Fourier approach can be used \citep[and are used; e.g.,][]{Portillo+2018,Philcox+2022ENCORE} to estimate or approximate $k$-point statistics in cosmology, which was our motivating question.

We leave the reader with the following questions. 
It is trivial to take the form of \eqref{eq:result} and evaluate the sums for more than three vectors; does the result have any meaning?
Is there a generalization of the triangle area \eqref{eq:result} and simplex volume \eqref{eq:simplex} results to a $d$-simplex?
Our formulae \eqref{eq:result} and \eqref{eq:simplex} only used invariant factors of degree $2$, with exponents that are larger than $2$. Does this generalize to higher dimension simplexes, and can it be shown that higher order factors aren't needed?
One can replace the sum in \eqref{eq:DeepSets} with any other permutation-invariant aggregator (e.g., max or product), though we do not know if there are any claims to universality in these cases. Does another area formula exist for these other aggregators?
We have discussed the general value of a map function $\phi$ which Fourier transforms the points;\footnote{%
\Hogg{The Deep Sets version of the Fourier transform is reminiscent of a positional encoding (CITE), similar to those used to preserve ordering information in (otherwise permutation invariant) transformers (CITE).}}
does a formula like \eqref{eq:EnergyDeepSets} exist for the area of a triangle?

{\footnotesize\par\bigskip\noindent
This project would not have been possible without originating inspiration and discussions with Soledad Villar (JHU).
The authors thank Ben Blum-Smith (JHU), Jonathan Goodman (NYU), Wilson Gregory (JHU), Teresa Huang (Flatiron), Mike O'Neil (NYU), Alexander Novara (NYU), Ben Wandelt (JHU), Kaze Wong (JHU), and the Blanton--Hogg group meeting at NYU for valuable discussions.
The authors also note that, while we believe the formula is (barely) novel, it is difficult to be sure, given the challenge of conducting a complete literature review for this ancient topic.\par}

\secbreak
% need a line break this time

\vspace{-\bigskipamount}
\renewcommand{\section}[2]{}%
{\small\singlespacing\bibliography{main}\par}

\secbreak
\appendix
\subsection{Appendix}
\label{sec:appendix}

\paragraph{Proof that area sums in quadrature} We prove equation \eqref{eq:quadrature}, that the area of a triangle in $d$ dimensions is the sum in quadrature of the areas of the triangles made by all orthogonal projections to two axes.

From the three position vectors $\vec{r}_1, \vec{r}_2, \vec{r}_3 \in \mathbb{R}^d$ (for $d \geq 2$), we consider the two displacement vectors $\vec{a} \equiv \vec{r}_2 - \vec{r}_1$ and $\vec{b} \equiv \vec{r}_3 - \vec{r}_1$.
Given these, we can write the two-dimensional triangle sub-area $\Delta_{jk}$ by first defining the projected two-dimensional displacement vectors $\vec{a}_{jk} = ( a_j, a_k )$, $\vec{b}_{jk} = ( b_j, b_k )$. Then,
\begin{equation}
    \Delta_{jk}
    = \frac{1}{2} \, \abs{ \vec{a}_{jk} \wedge \vec{b}_{jk} }
    = \frac{1}{2}
        \, \abs{a_j \, b_k - b_j \, a_k}~,
\end{equation}
using formula \eqref{eq:cross}.

Next, consider formula \eqref{eq:cross} with the full $d$-dimensional vectors:
\begin{align}
    \Delta^2
    &= \frac{1}{4}\, \abs{\vec{a} \wedge \vec{b}}^2 \nonumber\\
    &= \frac{1}{4}\, \abs{\vec{a}}^2 \, \abs{\vec{b}}^2 - \frac{1}{4}\, \abs{\vec{a} \cdot \vec{b}}^2 \nonumber\\
    &= \frac{1}{4} \sum_{1 \leq j, k \leq d}
        ( a_j \, a_j \, b_k \, b_k - a_j \, b_j \, a_k \, b_k ) \nonumber\\
    &= \frac{1}{4} \sum_{1 \leq j < k \leq d}
        ( a_j \, b_k - b_j \, a_k )^2 \nonumber\\
    &= \sum_{1 \leq j < k \leq d} \Delta_{jk}^2~.
\end{align}


\paragraph{Derivation of Deep Sets energy} Consider a system of $N$ point particles with masses $m_i$ and positions $\vec{r}_i$.
The energy density $\mathcal{E} = d U / d^3 x$ is given by
\begin{equation}
    \mathcal{E} = \frac{\rho \, \Phi,}{2}~,
\end{equation}
where $\rho$ is the mass density and $\Phi$ is the gravitational potential \citep{LandauLifshitzFields}.
These are related by the Poisson equation
\begin{equation}
    \nabla^2 \Phi = 4 \pi G \, \rho~,
\end{equation}
which implies the following relationship between their Fourier transforms
\begin{equation}
    -k^2 \, \mathcal{F}[\Phi] = 4 \pi G \, \mathcal{F}[\rho]~,
\end{equation}
with $k^2 \equiv \abs{\vec{k}}^2$.
Using the convolution theorem, we then have
\begin{align}
    \mathcal{F}[\mathcal{E}] 
    &= \frac{1}{2} \mathcal{F}[\rho \, \Phi] \nonumber\\
    &= \frac{1}{2} \, \frac{\mathcal{F}[\rho] \ast \mathcal{F}[\Phi]}{(2\pi)^3} \nonumber\\
    &= \frac{1}{2 \, (2\pi)^3} \, \mathcal{F}[\rho] \ast \left(
        -4 \pi G \, k^{-2} \, \mathcal{F}[\rho]
    \right)~.
    \label{eq:ConvolutionThm}
\end{align}
This form of the convolution theorem is proven below.
The Fourier transform of the mass density is
\begin{equation}
    \mathcal{F}[\rho](\vec{k})
    = \int \d^3 r
        \, e^{-i \, \vec{k} \cdot \vec{r}}
        \, \sum_{j=1}^{N}
        \, m_j
        \, \delta(\vec{r} - \vec{r}_j)
    = \sum_{j=1}^{N} \, m_j \, e^{-i \, \vec{k} \cdot \vec{r}_j}~,
\end{equation}
where we sum over $j$ and use $i=\sqrt{-1}$.
Note that the value of the Fourier transform at a wavenumber $\vec{k}$ is of the required form $\sum_i \phi(m_i, \vec{r}_i)$ for Deep Sets features.
Hence, the total energy is given in Deep Sets form by
\begin{align}
    \phi_{\vec{k}}(m_j, \vec{r}_j)
    &=m_j \, e^{-i \, \vec{k} \cdot \vec{r}_j}
    ~,
    \quad
    \sum_{i=1}^{N} \phi_{\vec{k}}(m_i, \vec{r}_i)
    = \mathcal{F}[\rho](\vec{k})~,
    \nonumber\\
    U \, \big( \sum_{i=1}^{N} \, \phi_{\vec{k}}(m_i, \vec{r}_i) \big)
    &= -\frac{G}{4 \pi^2} \,
        \int \d^3 r \,
        \mathcal{F}^{-1} \left[
            \mathcal{F}[\rho] \ast \left(
            k^{-2} \, \mathcal{F}[\rho] \right)
        \right]~.
\end{align}



\paragraph{Proof of the convolution theorem} We prove the claim (equation \eqref{eq:ConvolutionThm}) that
\begin{equation}
    \mathcal{F}[\rho \, \Phi] 
    = (2\pi)^{-3} \, \mathcal{F}[\rho] \ast \mathcal{F}[\Phi] ~.
\end{equation}
Note that we use the conventions:
\begin{align}
    \tilde{f}(\vec{k})
    = \mathcal{F}[f(\vec{x})]
    &= \int \d^3 x
        \, e^{-i \, \vec{k} \cdot \vec{x}}
        \, f(\vec{x})~, \\
    f(\vec{x})
    = \mathcal{F}^{-1}[\tilde{f}(\vec{k})]
    &= \int \frac{\d^3 k}{(2\pi)^3}
        \, e^{+i \, \vec{k} \cdot \vec{x}}
        \, f(\vec{x})~.
\end{align}
Consider $\mathcal{F}^{-1}[\tilde{\rho} \ast \tilde{\Phi}](\vec{x})$. 
\begin{align}
    \mathcal{F}^{-1}[\tilde{\rho} \ast \tilde{\Phi}](\vec{x})
    &= \int_{\mathbb{R}^3} \frac{\d^3 k}{(2\pi)^3}
        \, e^{i \, \vec{k} \cdot \vec{x}} 
        \, \big( \tilde{\rho} \ast \tilde{\Phi} \big) (\vec{k}) 
        \nonumber\\
    &= \int_{\mathbb{R}^3} \frac{\d^3 k}{(2\pi)^3}
        \, e^{i \, \vec{k} \cdot \vec{x}} 
        \, \int_{\mathbb{R}^3} \d^3 \vec{k}' 
        \, \tilde{\rho}(\vec{k}') 
        \, \tilde{\Phi}(\vec{k} - \vec{k}')
        \nonumber\\
    &= \int_{\mathbb{R}^3} \d^3 \vec{k}'
        \, \tilde{\rho}(\vec{k}') 
        \, \int_{\mathbb{R}^3} \frac{\d^3 k}{(2\pi)^3}
        \, e^{i \, \vec{k} \cdot \vec{x}} 
        \, \tilde{\Phi}(\vec{k} - \vec{k}')
        \nonumber\\
    &= \int_{\mathbb{R}^3} \d^3 \vec{k}'
        \, \tilde{\rho}(\vec{k}') 
        \, e^{i \, \vec{k}' \cdot \vec{x}}
        \, \Phi(\vec{x})
        \nonumber\\
    &= (2\pi)^3
        \, \rho(\vec{x}) 
        \, \Phi(\vec{x})~.
\end{align}
Taking the Fourier transform of both sides gives the desired result.


\end{document}
