% Note to self: Please don't work on this, ever.

% to-do items
% -----------
% - Show that there is NOT an exact solution at polynomial degree 2 order.
% - Make the code output something that could be executed by Mathematica.
% - Prove (maybe using Mathematica) that the formula reduces to Heron's formula.
% - Prove that the formula is exactly translation-invariant.
% - Prove that the formula is exactly rotation-invariant.
% - Clean the notation so $x$ is only ever used in one specific way.
% - State exactly the "rules of the game" in the introdution.

\documentclass[12pt]{article}
\usepackage[letterpaper]{geometry}
\usepackage{amsmath}

\newcommand{\documentname}{\textsl{Note}}
\newcommand{\sectionname}{Section}
\newcommand{\secref}[1]{\sectionname~\ref{#1}}
\frenchspacing

\begin{document}

\section*{\raggedright
Another formula for the area of a triangle: Useless, but explicitly in Deep-Sets and map--reduce form}
\noindent \textbf{David W. Hogg} and \textbf{Soledad Villar}

\paragraph{Abstract:}
It is known that any permutation-invariant function of data points $x_i$
can be written in the form $\rho(\sum_i\phi(x_i))$,
where $\rho$ and $\phi$ are (probably nonlinear) functions, and the function $\phi$ might output a high-dimensional latent encoding of each point $x_i$.
This form---known in the machine-learning literature as Deep Sets---is a map--reduce form, because the $\phi$ executions can be performed asynchronously and the sums can be performed hierarchically or in any order.
The area of a triangle is a permutation-invariant function of the locations of the three corners.
We find, possibly for the first time, the formula for the area of the triangle that is explicitly in the Deep-Sets form.
No insights of any kind were gained by this work.

\section{Introduction}
The area $A$ of a triangle can be written in many ways:
\begin{align}
A &= \frac{1}{2}\,\text{(base)}\,\text{(height)} \label{eq:school} \\
  &= \frac{1}{2}\,|(\vec{x}_3 - \vec{x}_1) \times (\vec{x}_2 - \vec{x}_1)| \label{eq:cross} \\
  &= \frac{1}{2}\,a\,b\,\sin(C) \label{eq:sine} \\
  &= \sqrt{s\,(s-a)\,(s-b)\,(s-c)} \label{eq:Heron} ~,
\end{align}
where the first formula is the usual introduction to the subject,
$(\vec{x}_1, \vec{x}_2, \vec{x}_3$ are three vectors pointing from an origin to the corners of the triangle,
$\times$ represents the cross-product operator (or its generalization to higher dimensions),
$a, b, c$ are side lengths of the triangle,
$C$ is the angle between the $a$ and $b$ sides,
and $s$ is the semi-perimeter $(a + b + c)/2$.
Below, we present a new formula for the area of a triangle, which we believe may be novel.

While the area of a triangle does not depend on how its sides or corners are labeled,
only the last of these formulae---Heron's formula \eqref{eq:Heron}---is explicitly permutation-invariant.

The theory of permutation-invariant functions says (CITE) that any per\-mu\-tation-invariant function (under conditions??) of $N$ objects $x_i$ (with $1\leq i\leq N$) can be approximated (to arbitrary precision??) with an expression of the form
\begin{equation}
    \rho(\sum_{i=1}^N\phi(x_i)) \label{eq:DeepSets}
\end{equation}
where $\rho$ and $\phi$ are (possibly very nonlinear) functions,
and the $x_i$ are the data points that can be permuted without changing the result.
This result---which we associate with the subject \emph{Deep Sets} (CITE)---is not at all obvious:
For example, even though Heron's formula \eqref{eq:Heron} is permutation-invariant, it is not actually in the form \eqref{eq:DeepSets}, if the triangle is seen as being defined by its corners (or even its sides).

A few comments about this Deep-Sets form \eqref{eq:DeepSets} we'd make are the following:
\begin{itemize}
    \item The function $\phi$ is universal to all the points $x_i$; it's form won't depend on any properties of each point, nor any properties of the overall collection of points (the \emph{point cloud}, as it were).
    \item The point properties $x_i$ and the output of the function $\phi$ might both be very large.
    That is, each point $x_i$ might be a vector or even an ordered list of vectors and tensors and images and so on, and the output of the function $\phi$ might also be very large and complicated.
    The output of the function $\phi$ is some kind of informative, latent encoding of the point $x_i$.
    \item If a function can be written in Deep-Sets form \eqref{eq:DeepSets}, then that function can be executed in a map--reduce framework, in which the function $\phi$ is executed independently and asynchronously on each data point (maybe even in the data center) and the sums are performed hierarchically or in any order.
    That means that for large problems executed on large data sets in large data centers, problems in this form \eqref{eq:DeepSets} can be performed in wall-clock time that scales as $\log N$.
\end{itemize}

Although the contribution of this paper is minimal, or even zero, it was inspired by real problems in physics.
For one very specific example, all known formulae for the gravitational potential energy of a system of $N$ point masses involve sums over pairs of points; they have not ever been written in the Deep Sets form \eqref{eq:DeepSets}.
In cosmology, all known estimators for $k$-point functions (correlation functions, bispectra, and so on) involve sums over $k$-tuples; once again they have not been written in the Deep Sets form.
Importantly, these kinds of problems cannot currently be performed (not even approximately) in any map--reduce framework; indeed they are very time-consuming because they involve sums over all elements of $N$-choose-$k$, which scales (badly) as $N^k$.
Because 3-point functions involve the shapes and areas of triangles, consideration of the 3-point function led to the question answered in this \documentname:
\emph{What is the Deep Sets (or map--reduce) expression for the area of a triangle?}
We deliver in \secref{sec:result}.

\section{A note on our method}\label{sec:method}
What made us think $A^2$ rather than $A$?

What do units tell you?

How to construct all the invariant polynomials.

Which polynomial terms are invariant to other kinds of transformations?

Symbolic regression, determination of the magic number, and checking.

\section{Result and discussion}\label{sec:result}
Our result---and only contribution---is that the area $A$ of a triangle is given by
\begin{align}
A^2 = &
-\frac{ 54}{368}\,(\sum_{i=1}^3 x_i^2\,y_i^2) 
-\frac{258}{368}\,(\sum_{i=1}^3 x_i\,y_i)^2 \nonumber \\ &
+\frac{285}{368}\,(\sum_{i=1}^3 x_i^2)\,(\sum_{i=1}^3 y_i^2) 
+\frac{ 36}{368}\,(\sum_{i=1}^3 y_i)\,(\sum_{i=1}^3 x_i^2\,y_i) \nonumber \\ &
+\frac{ 36}{368}\,(\sum_{i=1}^3 x_i)\,(\sum_{i=1}^3 x_i\,y_i^2) 
-\frac{101}{368}\,(\sum_{i=1}^3 y_i)^2\,(\sum_{i=1}^3 x_i^2) \nonumber \\ &
+\frac{148}{368}\,(\sum_{i=1}^3 x_i)\,(\sum_{i=1}^3 y_i)\,(\sum_{i=1}^3 x_i\,y_i) 
-\frac{101}{368}\,(\sum_{i=1}^3 x_i)^2\,(\sum_{i=1}^3 y_i^2) \nonumber \\ &
+\frac{  9}{368}\,(\sum_{i=1}^3 x_i)^2\,(\sum_{i=1}^3 y_i)^2 ~, \label{eq:result}
\end{align}
where the two-dimensional position of each of the three corners $i$ is $(x_i, y_i)$.
This answer is in Deep-Sets form:
Every expression in parentheses in \eqref{eq:result} is a pure sum over all three points.
There are no sums over differences, and there is no ordering of the points.
Being very explicit, this can be written in the Deep Sets form \eqref{eq:DeepSets} if we define $\rho$ and $\phi$ as follows:

Note the symmetries....! See the $36/368$ terms, or the $-101/368$ terms

But note that it isn't obviously invariant to arbitrary rotations.
And it isn't obviously invariant to translations.
Formulae \eqref{eq:cross}, \eqref{eq:sine}, and \eqref{eq:Heron} are obviously rotation-invariant and translation-invariant.

Why not simplify the prefactors?

What's the generalization to $d$ dimensions?

Where does the factor of 23 come from?

What happens if we try to do $A$ as a quadratic instead of $A^2$ as a quartic?

Why not get this with computer algebra?

\end{document}
